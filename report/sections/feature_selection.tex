\section{Feature selection}

Even if we run a PCA on the embeddings, we still have a lot of features in our dataset. We performed feature selection in order to decrease the number of variables in our dataset. We investigated multiple technique to select features: correlation, mutual information and recursive feature elimination. The problem with the first one is that it is only relevant for linear models since a low correlation does not mean the absence of relationship for non-linear models. We could have selected features highly correlated with the target for the linear regression but we found that even for this model we had slightly better results with mutual information.

We first tried to remove redundant features. The idea was to check the features that are highly correlated together (\textit{above a certain treshold}) and then keep the one that had the most mutual information with the target. However we didn't find any redundant features in our dataset (at least after the PCA).

\subsection{Mutual information}

Mutual information has the advantage over correlation to detect nonlinear relationships between variables and therefore is a filter of choice for nonlinear models. Our strategy was to minimize the redundancy between inputs variables and a maximize relevancy between inputs variables and the target variable (\textit{minimum redundancy, maximum relevance}). The idea is that even if $2$ features are highly relevant, we shouldn't add both of them to our subset of features if they are highly correlated since it would increase the model complexity and could cause overfitting.

To minimize the redundancy, we computed a normalized mutual information matrix and decided to remove for each group of redundant features, the feature that share the less information with the target.
Then to maximize the relevancy, we choosed to keep a certain percentage of most significant features (\textit{i.e. features that share the most information with target}).

\begin{figure}[H]
	\centering
	\includegraphics{figures/MI_with_target.tex}
	\caption{Mutual information of each feature with the target variable (revenues)}
	\label{fig:MI_with_target}
\end{figure}

We can notice that the most relevant features are \textit{studio}, \textit{n_votes} and \textit{release_year}. There approximately $50\%$ of features that do not share information with the target and that we can drop

\subsection{Wrapper method}

We also tried to implement a \textbf{recursive feature elimination} (RFE). The idea behind RFE is to start with all the features and then successively removing one feature at at time until reaching a desired number of features. However, we didn't kept this method since on one hand it is computationnaly intensive and on the other hand, it very dependent on the chosen model.

% TODO: finish 